% !TeX root = 6_3_48_A_two_win_B_finally_win.tex

% https://www.overleaf.com/read/fgtyhvhrgtkv#384ea2

% Use "pdftoppm Discrete_Mathematics_and_Its_Applications_6_3_48.pdf | pnmcrop | pnmtopng > doc.png" 
% to convert the pdf to the image. https://tex.stackexchange.com/questions/11866/compile-a-latex-document-into-a-png-image-thats-as-short-as-possible#comment141661_11868
% or use "file=6_3_48_A_two_win_B_finally_win;pdfcrop ${file}.pdf; pdftoppm ${file}-crop.pdf | pnmtopng > file.png"
% the latter is more stable.

% ppm is one type of pnm https://netpbm.sourceforge.net/doc/pnm.html

% \documentclass[convert={density=300,size=1080x800,outext=.png}]{standalone}
\documentclass{article}

% https://tex.stackexchange.com/a/384538/308105
\usepackage{amssymb}
\usepackage[dvipsnames,table]{xcolor}
\usepackage{pifont}
\pagenumbering{gobble}
% https://tex.stackexchange.com/a/42620/308105
\newcommand{\xmark}{\ding{55}}
\newcommand{\cmark}{\ding{51}}
\newcommand*\ColourTickCheck[1]{%
  \expandafter\newcommand\csname #1TickCheck\endcsname{\color{#1}\cmark}%
}
\newcommand*\ColourXCheck[1]{%
  \expandafter\newcommand\csname #1XCheck\endcsname{\color{#1}\xmark}%
}
% \definecolor{lightgreen}{RGB}{0,255,127}
% \definecolor{lightgreen}{RGB}{50,205,50}
\definecolor{lightgreen}{HTML}{5CF766}
\ColourTickCheck{lightgreen}
\ColourXCheck{red}

% https://tex.stackexchange.com/a/241016/308105
% \newcommand{\ShadowBackground}[1]{\colorbox{black!30}{#1}}

% https://www.overleaf.com/learn/latex/Tables#Colouring_a_table_(cells,_rows,_columns_and_lines)
\usepackage{colortbl}
\definecolor{Gray}{gray}{0.95}
\newcolumntype{a}{>{\columncolor{Gray}}m{0.8em}}

\begin{document}
% \section{6.4 48}
\begin{center}
\begin{tabular}[c]{ cclc } 
 \textbf{Team A} & \textbf{Team B} & \textbf{Penalty shoot-out} & \textbf{Combinations}\\
 \hline
 \((2,2)\) & \((4,0)\) & 
 \begin{tabular}{ aaal }
 \lightgreenTickCheck & \lightgreenTickCheck & \redXCheck & \cellcolor[gray]{0.95}\redXCheck\\
 \lightgreenTickCheck & \lightgreenTickCheck & \lightgreenTickCheck & \lightgreenTickCheck
 \end{tabular} & \(C(4,2)C(3,3)=6\)\\
 \hline
 \((2,3)\) & \((3,1)\) & 
 \begin{tabular}{ aaaal }
 \lightgreenTickCheck & \lightgreenTickCheck & \redXCheck & \redXCheck & \redXCheck\\
 \redXCheck & \lightgreenTickCheck & \lightgreenTickCheck & \lightgreenTickCheck
 \end{tabular} & \(C(4,2)C(4,3)=24\)\\
 \hline
 \((2,3)\) & \((3,2)\) & 
 \begin{tabular}{ aaaal }
 \lightgreenTickCheck & \lightgreenTickCheck & \redXCheck & \redXCheck & \cellcolor[gray]{0.95}\redXCheck\\
 \redXCheck & \redXCheck & \lightgreenTickCheck & \lightgreenTickCheck & \lightgreenTickCheck
 \end{tabular} & \(C(5,2)C(4,2)=60\)\\
 \hline
%  https://en.wikibooks.org/wiki/LaTeX/Rules_and_Struts https://tex.stackexchange.com/a/31704/308105
\rule{0pt}{1.1em}
 &&&Total=90
\end{tabular}
\end{center}

\end{document}

