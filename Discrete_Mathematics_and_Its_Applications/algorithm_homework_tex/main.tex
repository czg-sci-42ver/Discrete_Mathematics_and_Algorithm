% https://tex.stackexchange.com/a/145660/308105 https://latex.org/forum/viewtopic.php?p=51656#p51656 to use "\setlength{\mathindent}{0pt}"
\documentclass{article}
\usepackage{graphicx} 
\usepackage{amsmath}
\usepackage[noend,ruled,lined]{algorithm2e}
\usepackage{mathtools}
\usepackage{mfirstuc}
\usepackage{enumitem}
\usepackage{pifont,amssymb}
\usepackage{float}

\usepackage[normalem]{ulem}

\usepackage[nodisplayskipstretch]{setspace} % for "\setstretch" macro

% \usepackage[normalem]{ulem}

% https://www.overleaf.com/learn/latex/Hyperlinks#Reference_guide to use ref link
% https://tex.stackexchange.com/a/82913/308105
% TODO cleveref https://tex.stackexchange.com/questions/83037/difference-between-ref-varioref-and-cleveref-decision-for-a-thesis
\usepackage{hyperref}
% https://www.overleaf.com/learn/latex/Using_colours_in_LaTeX
\usepackage[dvipsnames]{xcolor}
\hypersetup{
    colorlinks=true,
    linkcolor=Green,
    filecolor=magenta,      
    urlcolor=cyan,
    pdftitle={Overleaf Example},
    pdfpagemode=FullScreen,
    }
% https://tex.stackexchange.com/a/39228/308105
\usepackage{indentfirst}

\makeatletter
\setlength{\@fptop}{0pt}
\setlength{\@fpsep}{3pt}
\setlength{\@fpbot}{0pt plus 1fil}
\makeatother

% https://www.overleaf.com/learn/latex/Page_size_and_margins#Paper_size,_orientation_and_margins
% \usepackage{layout}
% \setlength{\hoffset}{0pt}
% \setlength{\oddsidemargin}{0pt}
% \setlength{\marginparsep}{0pt}
% \setlength{\textwidth}{500pt}
% https://tex.stackexchange.com/a/367867/308105
\usepackage[left=0.3in,textwidth=7.3in,top=0.3in,bottom=0.3in]{geometry}

\newlist{todolist}{enumerate}{2}
% https://www.tug.org/TUGboat/tb09-1/tb20bechtolsheim.pdf
\setlist[todolist]{label={\arabic*.\makebox[0pt][r]{\noexpand\emptysquare\hspace{1.3em}}},ref=\arabic*}

% https://tex.stackexchange.com/a/138996/308105
\usepackage{etoolbox,setspace}
% \AtBeginEnvironment{algorithm}{\setstretch{20.5}}
\setstretch{1.2}

\newcommand{\emptysquare}{$\square$}
\newcommand{\checkedsquare}{\makebox[0pt][l]{\raisebox{1pt}[0pt][0pt]{\large\hspace{1pt}\cmark}}$\square$}
\newcommand{\cmark}{\ding{51}}%
\newcommand{\correctanswer}{{\renewcommand{\emptysquare}{\checkedsquare}\item\leavevmode}}

% https://tex.stackexchange.com/a/20613/308105
% here 0 is one renaming for one box https://tex.stackexchange.com/a/46483/308105
% \rule https://en.wikibooks.org/wiki/LaTeX/Rules_and_Struts
% \wd https://tex.stackexchange.com/a/222617/308105
% \rlap https://www.tutorialspoint.com/tex_commands/rlap.htm https://tex.stackexchange.com/a/146100/308105
% \ht https://tex.stackexchange.com/a/258779/308105
\newcommand\hcancel[2][black]{\setbox0=\hbox{$#2$}%
\rlap{\raisebox{.45\ht0}{\textcolor{#1}{\rule{\wd0}{1pt}}}}{#2}} 

% \setlength{\mathindent}{0pt}

% https://www.overleaf.com/learn/latex/%5Chfuzz this doesn't influence the typesetting https://tex.stackexchange.com/a/50850/308105
\hfuzz=50pt

% https://tex.stackexchange.com/a/169711/308105
% \renewcommand{\topfraction}{.5}
% \renewcommand{\floatpagefraction}{.75}
% https://www.overleaf.com/learn/latex/Environments#Defining_simple_environments and https://tex.stackexchange.com/a/67849/308105
\newenvironment{SpacingAlgorithm}[2]{
% this will make the figures more condensed https://tex.stackexchange.com/a/35130/308105
% \nopagebreak may not influence this
% TODO see this https://tex.stackexchange.com/a/2282/308105, https://tex.stackexchange.com/questions/701558/how-to-make-the-multiple-floating-blocks-not-occupying-the-whole-page-while-keep?noredirect=1#comment1744632_701558 to think of one better solution
% with the \par, then no need for the preceding "\\" to separate it from others.
    \par\addvspace{10pt}
    \begin{algorithm}[H]
    % \begin{algorithm}[htb]
        \setstretch{1.10}
        \caption{#1}\label{alg:#2}
}
{
% https://tex.stackexchange.com/a/701544/308105
    \end{algorithm}
    \vspace{10pt}
}
\setlist{nolistsep}

% https://tex.stackexchange.com/q/701531/308105
\makeatletter
\newenvironment{shiftedflalign*}{%
    \vspace{-10pt}
    \start@align\tw@\st@rredtrue\m@ne
    % \hspace{-7pt}
}{%
    \endalign
    \vspace{-40pt}
}
\makeatother

% https://tex.stackexchange.com/a/701546/308105
\newenvironment{shiftedalign*}[1][5pt]{%
  \par\vspace{0ex}% adjust
  % \hspace{#1}%
  $\begin{aligned}%
}{%
  \end{aligned}$
  \par\vspace{+1ex}% adjust
}

% https://tex.stackexchange.com/a/234320/308105
% here should use \def https://tex.stackexchange.com/a/262/308105
\edef\mylb{\\}
% \def\mylb{\newline}
\newcommand{\CondensedNewline}{\mylb[-0.5em]}

% https://tex.stackexchange.com/questions/518403/how-to-use-newenvironment-and-align-in-standalone-document#comment1311023_518403
\newenvironment{CondensedAlign}{%

% % the \\ redefinition will make \begin{cases} fail which is weird.
% \let\mylb\\
% \renewcommand{\\}{\mylb[-0.1cm]}

% % https://tex.stackexchange.com/a/234320/308105 TODO why this only takes effects in align
% \let\myhalign\halign
% \def\halign{%
% \let\mylb\\
% \renewcommand{\\}{\mylb[-0.1cm]}
% \myhalign
% }

% only valid for cases* by the mathtools package which also allows for & https://tex.stackexchange.com/a/14680/308105
\addtolength{\jot}{-5pt}

  \par\vspace{0ex}% adjust
  $\displaystyle\aligned
}{
  \endaligned$
  \par\vspace{+1ex}% adjust
  \renewcommand{\\}{\mylb}
}

\newenvironment{CondensedCases}{%

% these don't work for the cases block
% \let\mylb\\
% \renewcommand{\\}{\mylb[1cm]}

% \let\myhalign\hhalign
% % notice here def can't be same as above halign, otherwise many weird errors.
% \gdef\hhalign{%
% \let\mylb\\
% \renewcommand{\\}{\mylb[40cm]}
% \myhalign
% }

% https://tex.stackexchange.com/a/472518/308105 doesn't work for the inline mode, although warning "Command \normalsize invalid in math mode.".
\everydisplay=\expandafter{\the\everydisplay\setstretch{0.9}}

  \par\vspace{-5ex}% adjust
  % $\displaystyle\begin{cases}
  $$\displaystyle\begin{cases}
  % $\displaystyle\aligned
  % \begin{cases}
}{
  %   \end{cases}
  % \endaligned$
  \end{cases}$$
  \par\vspace{-2ex}% adjust
  \renewcommand{\\}{\mylb}
}

% https://tex.stackexchange.com/a/228609/308105 this can't work
%\let\CondensedNewline\vspace{0.5em}
% https://stackoverflow.com/q/43821452/21294350 not work
% \newcommand{\CondenseNewline}{\newline[-1em]}

% https://tex.stackexchange.com/questions/701531/how-to-tune-the-multiline-equation-with-the-self-defined-indentation?noredirect=1#comment1744479_701531
\newcommand{\NoIndentEquationLine}[1]{%
\hspace{-5pt}#1&&\CondensedNewline%
}

\newcommand{\mycmt}[1]{
    \{#1\}
}
% https://tex.stackexchange.com/a/451651/308105
\newcommand{\mywrongitem}{\itemsep0em \addtocounter{todolisti}{1}\item}
\newcommand{\myrightitem}{\itemsep0em \addtocounter{todolisti}{1}\correctanswer}

% must have one blank line between it and the "\end{algorithm}"
\newcommand{\ecc}[1]{%
    see the ans:%
    \begin{itemize}[noitemsep]%
        #1%
    \end{itemize}
}

\newcommand{\ProcTargetName}{test}
\newcommand{\NormalProcTargetName}{test}
\newcommand{\CapProcTargetName}{\capitalisewords{\ProcTargetName}}
\newcommand{\AlgorithmRef}{}
% https://tex.stackexchange.com/a/100930/308105
\newcommand{\CurTodolistToplevelIndex}{\number\numexpr\value{todolisti}\relax}
% \renewcommand{\CurTodolistToplevelIndex}{\number\numexpr\value{todolisti}\relax}

% use one newcommand to implement 2 \newcommand otherwise the latter one can't reuse the former's renewcommand https://tex.stackexchange.com/a/259292/308105
% test with \textbf this is the \capitalisewords problem.
\newcommand{\ModProcName}[1]{
    \renewcommand{\ProcTargetName}{\textit{#1}}
    \renewcommand{\NormalProcTargetName}{#1}
    % here should not have maths codes with space inside "$(x, y, z)$ with $(y, z, x)$" which will be ambiguous.
    \renewcommand{\CapProcTargetName}{\capitalisewords{#1}.}
    
    % \def{\ProcTargetName}{\textit{#1}}

    % \def can be nested but it is no use maybe for \capitalisewords https://tex.stackexchange.com/a/318/308105 
    % \def\NormalProcTargetName{#1}
    % \def\CapProcTargetName{\capitalisewords{\NormalProcTargetName}.}
    
    % https://tex.stackexchange.com/a/70153/308105
    \renewcommand{\AlgorithmRef}{\text{ See }\hyperref[alg:\CurTodolistToplevelIndex]{this}}
}

\SetKwComment{Comment}{$\{$ }{ $\}$}
\SetKwInOut{Input}{input}
\SetKwInOut{Output}{output}
% https://tex.stackexchange.com/a/302537/308105 or https://tex.stackexchange.com/a/271771/308105 which uses something like \upshape https://tug.org/pracjourn/2006-1/schmidt/schmidt.pdf
% all default formats https://tex.stackexchange.com/a/623663/308105
\SetProgSty{textnormal}

% change for all https://tex.stackexchange.com/a/279202/308105
% \AtBeginEnvironment{algorithm}{\let\textnormal\ttfamily}

% https://stackoverflow.com/a/2212644/21294350
\SetKwProg{proc}{procedure}{\vspace{-5pt}}{}
% 
\newcommand{\myproc}[2]{\proc{\normalfont#1}{#2}}
\SetKwFor{While}{while (}{)}{}
\SetKwFor{For}{for }{}{}
\SetKw{Continue}{continue}
\SetKw{Break}{break}
% \SetKwIF{If}{ElseIf}{Else}{if}{then}{else if}{else}{\vspace{4pt}}
% https://tex.stackexchange.com/a/47353/308105
\let\oldelse\Else
\renewcommand{\Else}[1]{\oldelse{#1\vspace{4pt}}}
\newcommand{\ForRange}[4]{%
    % https://www.overleaf.com/learn/latex/Spacing_in_math_mode
    \For{#1\,$\coloneqq$\,\text{\normalfont #2}\,\KwTo \text{#3}}{ #4}
}
\DontPrintSemicolon
% \renewcommand{\Return}[1]{\vspace{4pt}\KwRet{#1}}

% https://tex.stackexchange.com/a/639572/308105
% left aligned https://tex.stackexchange.com/a/145660/08105
% \newenvironment{AlignNoSpacing}{
%     \vspace{-15pt}
%     \begin{flalign*}
% }{
%     \end{flalign*}
%     \vspace{-25pt}
% }

\newenvironment{absolutelynopagebreak}
  {\par\nobreak\vfil\penalty0\vfilneg
   \vtop\bgroup}
  {\par\xdef\tpd{\the\prevdepth}\egroup
   \prevdepth=\tpd}

\begin{document}

\setlength{\parindent}{20pt}
\section{TODO}
\subsection{use codes to check the following exercises}
% \renewcommand{\\}{\CondensedNewline}
% \let\mylb\\
% \renewcommand{\\}{\mylb[0.1cm]}
test \\
test
% \renewcommand{\\}{\mylb}
\begin{itemize}
    \item 3.1-28,
\end{itemize}
\section{Predefined Rules}
\begin{enumerate}
    \item If the following item has: \\
          \ecc{\item }
          Then probably this item has errors corrected by the ans(wer).

          % not use verbatim* which contains visible whitespaces as short squat "u"s https://tex.stackexchange.com/a/448105/308105
          % https://tex.stackexchange.com/a/107657/308105
    \item for the \verb|\begin{align*}| and \verb|\begin{cases*}|, I \sout{use the same format as the book p206 shows} made them condensed to be better viewed on one page together.
\end{enumerate}
\section{section 3.1}

% \texttt{\begin{align*} and \begin{cases*}}

% https://tex.stackexchange.com/a/64476/308105
\setlength{\textfloatsep}{0pt}% Remove \textfloatsep
\setlength{\intextsep}{4pt}% Remove \textfloatsep
% https://www.overleaf.com/learn/latex/Font_sizes%2C_families%2C_and_styles#Reference_guide
\begin{large}
    14 skipped
\end{large}

\begin{todolist}[itemsep=0pt,parsep=0pt,start=3]
    \ModProcName{largest difference}
    % \ProgSty{text}\\
    \myrightitem \CapProcTargetName \AlgorithmRef
    % \CurTodolistToplevelIndex
    \begin{SpacingAlgorithm}{\CapProcTargetName}{\CurTodolistToplevelIndex}
        \myproc{\ProcTargetName($a_1, a_2,\ldots , a_n: \text{integers})$}{
        }
        $i\coloneqq 1\mycmt{i\text{ is the search index}}$\\
        $d\coloneqq 0\mycmt{d\text{ is the target difference}}$\\
        \While{$i\le n-1$}{
        \lIf{$d<a_{i+1}-a_i$}{$d=a_{i+1}-a_i$}
        }
        \Return{d} \Comment{test}
    \end{SpacingAlgorithm}

    \ModProcName{number of negative integers}
    \mywrongitem \CapProcTargetName \AlgorithmRef
    \begin{SpacingAlgorithm}{\CapProcTargetName}{\CurTodolistToplevelIndex}
        \myproc{\ProcTargetName($a_1, a_2,\ldots , a_n: \text{integers})$}{}
        $i\coloneqq 1\mycmt{i\text{ is the search index}}$\\
        $c\coloneqq 0\mycmt{c\text{ is the number of negative integers}}$\\
        \While{$i\le n$}{
            \lIf{$a_i<0$}{$c=c+1$}
        }
        \Return{c}
    \end{SpacingAlgorithm}

    see the ans:
    \begin{itemize}[noitemsep]
        \item here the step size of $i$ is not explicitly shown.
    \end{itemize}

    \ModProcName{location of the largest even integer}
    \mywrongitem \CapProcTargetName \AlgorithmRef
    \begin{SpacingAlgorithm}{\CapProcTargetName}{\CurTodolistToplevelIndex}
        \myproc{\ProcTargetName($a_1, a_2,\ldots , a_n: \text{integers})$}{}
        $c\coloneqq 1\mycmt{c\text{ is the location index}}$\\
        $find\_even\coloneqq 0$\mycmt{$find\_even$ \text{ tracks whether even number exists}}\\
        \ForRange{i}{1}{n}{
            \lIf{$(a_i\%2==0$\text{ \normalfont and }$a_c<a_i)$}{$c=i$}
        }
        \Return{$find\_even$}
    \end{SpacingAlgorithm}

    \ecc{
        \item here should initialize the comparison with $-\infty$
        \item $find\_even$ isn't assigned value in the loop
    }

    \ModProcName{compute $x^n$}
    \mywrongitem \CapProcTargetName \AlgorithmRef
    \begin{SpacingAlgorithm}{\CapProcTargetName}{\CurTodolistToplevelIndex}
        \myproc{\ProcTargetName($x$: real number, $n$: an integer)}{}
        $c\coloneqq 1\mycmt{c\text{ is the target product}}$\\
        \If{$n\ge 0$}{
            \ForRange{i}{1}{n}{
                $c=c*\cdot x$
            }
        }
        \Else{
            $n=-n$\\
            use this new $n$ to calculate\\
        }
        \Return{$c$}
    \end{SpacingAlgorithm}

    \ecc{
        \item use the absolute to combine the 2 cases
        \item the above lack the reciprocal for the negative case.
    }

    \ModProcName{replaces the triple $(x,y,z)$ with $(y,z,x)$}
    \myrightitem \CapProcTargetName \AlgorithmRef\\
    Here obviously at least 3, but this will lose one variable when assigning, so one extra to save this lost one.
    \begin{SpacingAlgorithm}{\CapProcTargetName}{\CurTodolistToplevelIndex}
        \myproc{\ProcTargetName($(x, y, z)$: target tuple)}{}
        $c\coloneqq 0\mycmt{c\text{ is the tmp variable}}$
        % https://tex.stackexchange.com/a/259088/308105
        % \begin{shiftedflalign*}
        %   \NoIndentEquationLine{c=x}
        %   \NoIndentEquationLine{x=y}
        %   \NoIndentEquationLine{c=x}
        %   \NoIndentEquationLine{c=x}
        % \end{shiftedflalign*}

        \begin{shiftedalign*}
            c=x\CondensedNewline
            x=y\CondensedNewline
            y=z\CondensedNewline
            z=c\CondensedNewline
        \end{shiftedalign*}

        \Return{$(x,y,z)$}
    \end{SpacingAlgorithm}
    \addtocounter{todolisti}{2}
    \ModProcName{smallest integer in a finite sequence}
    \mywrongitem \CapProcTargetName \AlgorithmRef
    \begin{SpacingAlgorithm}{\CapProcTargetName}{\CurTodolistToplevelIndex}
        \myproc{\ProcTargetName($a_1, a_2,\ldots , a_n: \text{the sequence}$)}{}
        $c\coloneqq \infty\mycmt{c\text{ is the target number}}$\\
        \ForRange{i}{1}{n}{
            \If{$c>a_i$}{
                $c=a_i$
            }
        }

        \Return{$c$}
    \end{SpacingAlgorithm}

    \ecc{
        \item $c=a_1$ is duplicate.
    }

    % \newpage
    \ModProcName{the last occurrence of the smallest element }
    \myrightitem \CapProcTargetName \AlgorithmRef\\
    This is similar to 16
    \begin{SpacingAlgorithm}{\CapProcTargetName}{\CurTodolistToplevelIndex}
        \myproc{\ProcTargetName($a_1, a_2,\ldots , a_n: \text{the sequence}$)}{}
        $c\coloneqq a_1\mycmt{c\text{ is the target number}}$\\
        $l\coloneqq 1\mycmt{l\text{ is the target number index}}$\\
        \ForRange{i}{2}{n}{
            \If{$c\ge a_i$}{
                \vspace{4pt}
                $c=a_i$\\
                $l=i$
                \vspace{4pt}
            }
        }

        \Return{$l$}
    \end{SpacingAlgorithm}

    \ModProcName{ the largest and the smallest integers}
    \mywrongitem \CapProcTargetName \AlgorithmRef
    \begin{SpacingAlgorithm}{\CapProcTargetName}{\CurTodolistToplevelIndex}
        \myproc{\ProcTargetName($a_1, a_2,\ldots , a_n: \text{the sequence}$)}{}
        $c\coloneqq a_1\mycmt{c\text{ is the target smallest number}}$\\
        $l\coloneqq a_1\mycmt{l\text{ is the target largest number}}$\\
        \ForRange{i}{2}{n}{
            \lIf{$c\ge a_i$}{
                $c=a_i$
            }
            \lIf{$l\le a_i$}{
                $l=a_i$
            }
        }

        \Return{$c,l$}
    \end{SpacingAlgorithm}

    \ecc{
        \item different from 18, here we don't need to track \textbf{equal} condition.
    }

    \ModProcName{ the longest word in an English sentence}
    \myrightitem \CapProcTargetName \AlgorithmRef
    % \clearpage
    \begin{SpacingAlgorithm}{\CapProcTargetName}{\CurTodolistToplevelIndex}
        \myproc{\ProcTargetName($a_1, a_2,\ldots , a_n: \text{symbols}$)}{}
        $c\coloneqq 0\mycmt{c\text{ is the track length}}$\\
        $len\coloneqq 0\mycmt{len\text{ is the max length}}$\\
        $word\coloneqq blank\mycmt{word\text{ is the finded word}}$\\
        \ForRange{i}{1}{n}{
        \lIf{$a_i\text{ is a letter}$}{
            $c=c+1$
        }
        \ElseIf{$a_i\text{ is a blank}$}{
        $c=0$\\
        \If{$len<c$}{
        $len=c$\\
        $word\text{ is }substr(a_{i-len}:a_{i-1})$
        }
        }
        \lElse{
            error
        }
        }

        \Return{$word$}
    \end{SpacingAlgorithm}

    Here I use $substr$ which is similar to one predefined function in C, like \href{https://stackoverflow.com/a/4214350/21294350}{this} or others, to replace the ans $word\coloneqq\lambda\ldots word\coloneqq concatenation\ldots$\\
    Also, $length\coloneqq0$ is implied in the above $c=0$.\\
    The ans uses $\lambda$ to represent empty string as book p189 shows.
    \ModProcName{ whether a function is one-to-one}
    \mywrongitem \CapProcTargetName \AlgorithmRef\\
    \ecc{
        \item lack the func parameter
        \item My original thoughts are that we put all values mapped from the domain in one new set, and check whether duplicate conditions occur for each mapped value by \textbf{traversing} the list for the same value with the current calculated one. Obviously, this has the much higher complexity.\\
        The ans just use the $hit$ to make $one\_one$ false when accessed twice, avoiding the traverse.
    }
    \begin{SpacingAlgorithm}{\CapProcTargetName}{\CurTodolistToplevelIndex}
        \myproc{\ProcTargetName($\{a_1, a_2,\ldots , a_n\}: \text{function domain},\{b_1, b_2,\ldots , b_n\}: \text{function codomain}$)}{}
    \end{SpacingAlgorithm}
    \ModProcName{ terminating Algorithm 3 if $x=a_m$}
    \mywrongitem \CapProcTargetName\\
    add $\text{if }x==a_m\text{ then }find=1;location=i;break$ before the 1st if block, \\
    then $\text{if }find==1\text{ then }return\ldots$ before the 2nd if block\\
    \ecc{
        \item the ans modified the action when $x<a_m$, this speeds up more.
    }
    \ModProcName{four sublists}
    \mywrongitem \CapProcTargetName \AlgorithmRef
    \begin{SpacingAlgorithm}{\CapProcTargetName}{\CurTodolistToplevelIndex}
        \myproc{\ProcTargetName($\{a_1, a_2,\ldots , a_n\}: \text{a list},x:\text{ the target element}$)}{}
        $c_i\coloneqq 1\mycmt{c\text{ is the }i\text{th split point}}$\\
        $i\coloneqq 1,j\coloneqq n\mycmt{j,i\text{ is the tmp index}}$\\
        $location\coloneqq 0\mycmt{location\text{ is the target index, which is inited "not found"}}$\\
        \While{$i<j$}{
            $c_1=i+\lfloor\frac{j-i}{4}\rfloor$\\
            $c_2=i+\lfloor\frac{j-i}{2}\rfloor$\\
            $c_3=i+\lfloor\frac{(j-i)*3}{4}\rfloor$\\
            \lIf{$x\le a_{c_1}$}{
                $j=c_1$
            }
            \ElseIf{$x\le a_{c_2}$}{
                $i=c_1$\\
                $j=c_2$\\
            }\ElseIf{$x\le a_{c_3}$}{
                $i=c_2$\\
                $j=c_3$
            }\lElse{
                $i=c_3$
            }
        }
        \lIf{$x==a_j$}{
            $location=j$
        }
        \Return{$location$}
    \end{SpacingAlgorithm}

    \ecc{
        \item the ans compares with the middle first, and then to the left or right which avoids some comparison overheads.
        \item also when $x>a_{c_1}$, we should let $i=c_1+1$ to avoid the duplicate future comparison.
        \item TODO the ans seems to have one error because $\lfloor\frac{j+i}{4}\rfloor>i$ doesn't always hold.
    }
    \clearpage

    \ModProcName{four sublists corrected}
    \begin{SpacingAlgorithm}{\CapProcTargetName}{\CurTodolistToplevelIndex_corrected}
        \myproc{\ProcTargetName($\{a_1, a_2,\ldots , a_n\}: \text{a list},x:\text{ the target element}$)}{}
        $c_i\coloneqq 1\mycmt{c\text{ is the }i\text{th split point}}$\\
        $i\coloneqq 1,j\coloneqq n\mycmt{j,i\text{ is the tmp index}}$\\
        $location\coloneqq 0\mycmt{location\text{ is the target index, which is inited "not found"}}$\\
        \While{$i<j$}{
            $c_1=i+\lfloor\frac{j-i}{4}\rfloor$\\
            $c_2=i+\lfloor\frac{j-i}{2}\rfloor$\\
            $c_3=i+\lfloor\frac{(j-i)*3}{4}\rfloor$\\
            \If{$x\le a_{c_2}$}{
                \lIf{$x\le a_{c_1}$}{
                    $j=c_1$
                }\Else{
                    $i=c_1+1$\\
                    $j=c_2$\\
                }
            }
            \ElseIf{$x\le a_{c_3}$}{
                $i=c_2+1$\\
                $j=c_3$
            }\lElse{
                $i=c_3+1$
            }
        }
        \lIf{$x==a_j$}{
            $location=j$
        }
        \Return{$location$}
    \end{SpacingAlgorithm}


    \begin{itemize}
        \item Here I prove $i>j$ when out of the \textbf{while} loop is impossible.\\
              $\text{let }j-i=4n+k,k=0\sim 3$
              \begin{CondensedAlign}
                  c_1 & =n      \\
                  c_2 & =
                  \begin{cases*}
                      2n,   & k=0,1 \\
                      2n+1, & k=2,3
                  \end{cases*}\\
                  c_3 & =
                  \begin{cases*}
                      3n,   & k=0,1 \\
                      3n+1, & k=2   \\
                      3n+2, & k=3
                  \end{cases*}
              \end{CondensedAlign}
              To let $i>j$, here assume $i=0$
              although it is impossible based on the positive index assumption,
              but $c_i$ is only based on $j-i$, so it is ok to let $i=0$.
              % https://tex.stackexchange.com/a/203059/308105 here \begin{align*}
              % \begin{equation*}
              %     \begin{aligned}
              %     \end{aligned}
              % \end{equation*}

              % here cases* has no & available. https://tex.stackexchange.com/q/124751/308105

              %   $\begin{cases}
              \begin{CondensedCases}
                  1. & j=c_1,i=0\Rightarrow 0>n\text{ \textbf{impossible}}                                                                          \\
                  2. & j=c_2,i=c_1+1\Rightarrow \text{only }n+1>2n\text{ is possible}
                  \Rightarrow n=0\xRightarrow{i<j}i=0,j=1                                                                                           \\
                  & \Rightarrow x\le (a_{c_2}=a_0)\wedge x>(a_{c_1}=a_0)\text{ \textbf{impossible}}                                              \\
                  3. & j=c_3,i=c_2+1                                                                                                                \\
                  & \Rightarrow\begin{cases}
                      (k=0\text{ or }1) & 2n+1>3n\text{, here } x>(a_{c_2}=a_{c_3})(x>a_0)\text{.}              \\
                                        & \text{so }i=c_3+1=1\text{ \textbf{impossible} here}                   \\
                      (k=2)             & \text{\textbf{impossible} similarly to the above }2n+2>3n+1,i=c_3+1=2 \\
                      (k=3)             & 2n+2>3n+2\text{ trivially \textbf{impossible}}
                  \end{cases}                     \\
                  4. & i=c_3+1,j=4n+k\;\hcancel{\Rightarrow 3n+3>4n+3,\text{ \textbf{impossible}.}}                                                 \\
                  & \Rightarrow 3n+k-1+1>4n+k\text{ \textbf{impossible}.}\text{ only } 3n+k+1>4n+k\text{ is possible}                            \\
                  & \Rightarrow k=0,n=0,\text{originally }j-i=4n+k=0(\text{\textbf{also impossible} because \textbf{while} block condition }i<j)
                  %   \end{cases}$
              \end{CondensedCases}

              Based on the above, if $i<j$ into the while block,
              then in the next iteration of the while loop, $i>j$ is impossible.$\quad\blacksquare$
    \end{itemize}

    \ModProcName{four sublists almost same the ans}
    \begin{SpacingAlgorithm}{\CapProcTargetName}{\CurTodolistToplevelIndex_ans}
        \myproc{\ProcTargetName($\{a_1, a_2,\ldots , a_n\}: \text{a list},x:\text{ the target element}$)}{}
        $c_i\coloneqq 1\mycmt{c\text{ is the }i\text{th split point}}$\\
        $i\coloneqq 1,j\coloneqq n\mycmt{j,i\text{ is the tmp index}}$\\
        $location\coloneqq 0\mycmt{location\text{ is the target index, which is inited "not found"}}$\\
        \While{$i<j$}{
            $c_1=i+\lfloor\frac{j-i}{4}\rfloor$\\
            $c_2=i+\lfloor\frac{j-i}{2}\rfloor$\\
            $c_3=i+\lfloor\frac{(j-i)*3}{4}\rfloor$\\
            \If{$x> a_{c_2}$}{
                \If{$x\le a_{c_3}$}{
                    $i=c_2+1$\\
                    $j=c_3$
                }\Else{
                    $i=c_3+1$
                }
            }
            \ElseIf{$x> a_{c_3}$}{
                $i=c_1+1$\\
                $j=c_2$\\
            }\lElse{
                $j=c_1$
            }
        }
        \lIf{$x==a_j$}{
            $location=j$
        }
        \Return{$location$}
    \end{SpacingAlgorithm}
    similar to the above,
    \begin{CondensedCases}
        1. & j=c_3,i=c_2+1                                                                                                                     \\
        & \Rightarrow\begin{cases}
            (k=0\text{ or }1) & 2n+1>3n,n=0\text{, here } x>(a_{c_2}=a_0)\wedge x\le (a_{c_3}=a_0)\text{ \textbf{impossible}.} \\
            (k=2)             & \text{\textbf{impossible} similarly to the above }2n+2>3n+1,n=0,c_2=c_3=1                      \\
            (k=3)             & 2n+2>3n+2\text{ trivially \textbf{impossible}}
        \end{cases} \\
        2. & i=c_3+1,j=4n+k                                                                                                                    \\
        & \Rightarrow 3n+k-1+1>4n+k\text{ \textbf{impossible}.}\text{ only } 3n+k+1>4n+k\text{ is possible}                                 \\
        & \Rightarrow k=0,n=0,\text{originally }j-i=4n+k=0(\text{also \textbf{impossible} because \textbf{while} block condition }i<j)      \\
        3. & j=c_2,i=c_1+1\Rightarrow \text{only }n+1>2n\text{ is possible}
        \Rightarrow n=0 \xRightarrow{x>(a_{c_2}=a_{c_3}=a_0)} j=c_1\text{ \textbf{impossible}}                                                 \\
        4. & j=c_1,i=0\Rightarrow 0>n\text{ \textbf{impossible}}                                                                               \\
    \end{CondensedCases}

    \ModProcName{finds all modes appearing the maximum
        number of times}
    \mywrongitem \CapProcTargetName \AlgorithmRef
    \begin{SpacingAlgorithm}{\CapProcTargetName}{\CurTodolistToplevelIndex}
        \myproc{\ProcTargetName($\{a_1, a_2,\ldots , a_n\}: \text{a list}$)}{}
        $S\coloneqq \varnothing\mycmt{S\text{ is the set of modes}}$\\
        $c\coloneqq 0\mycmt{c\text{ is the count}}$\\
        $tc\coloneqq 0\mycmt{tc\text{ is the local count}}$\\
        \ForRange{i}{1}{n}{
            $tc=0$\\
            \ForRange{j}{i+1}{n}{
                \If{$a_j=a_i$}{
                    $tc=tc+1$
                }\Else{
                    \Break
                }
            }
            $i=j-1$\Comment{ here I use C-style for loop, which will plus one when into the next iteration, so I minus one here}
            \lIf{$c<tc$}{$\text{clear }S\text{ and then append }S\text{ with }a_i$}
            \lElseIf{$c=tc$}{
                $\text{append }S\text{ with }a_i$
            }
        }
        \Return{$S$}
    \end{SpacingAlgorithm}
    \ecc{
        \item here $tc$ tracks duplicate count because it is inited to $0$ before comparing $a_{i+1}$ with $a_i$
        \item I missed the assignment $c=tc$ after $c<tc$.
    }

    \ModProcName{find the two that are
        closest}
    \mywrongitem \CapProcTargetName \AlgorithmRef
    \begin{enumerate}[label=(\alph*)]
        \item \begin{SpacingAlgorithm}{\CapProcTargetName}{\CurTodolistToplevelIndex}
                  \myproc{\ProcTargetName($\{x_1, x_2,\ldots , x_n\}: \text{a list}$)}{}
                  $d\coloneqq 0\mycmt{d\text{ is the minimum distance}}$\\
                  \ForRange{i}{1}{n}{
                      \ForRange{j}{i+1}{n}{
                          \lIf{$d<|a_i-a_j|$}{
                              $d=|a_i-a_j|$
                          }
                      }
                  }
                  \Return{$d$}
              \end{SpacingAlgorithm}
        \item \begin{SpacingAlgorithm}{\CapProcTargetName}{\CurTodolistToplevelIndex_b}
                  \myproc{\ProcTargetName($\{x_1, x_2,\ldots , x_n\}: \text{a sorted non-decreasing list}$)}{}
                  $d\coloneqq 0\mycmt{d\text{ is the minimum distance}}$\\
                  \ForRange{i}{1}{n-1}{
                  \lIf{$d<a_{i+1}-a_i$}{
                  $d=a_{i+1}-a_i$
                  }
                  }
                  \Return{$d$}
              \end{SpacingAlgorithm}
    \end{enumerate}
    \ecc{
        \item the question needs the pair
        \item better initing $d\coloneqq|x_1-x_2|$ to avoid the compulsory first assignment.
        \item it finds minimum instead of maximum.
    }
    \begin{SpacingAlgorithm}{\capitalisewords{hinted by the ans}, \CapProcTargetName}{\CurTodolistToplevelIndex_a_mod}
        \myproc{\ProcTargetName($\{x_1, x_2,\ldots , x_n\}: \text{a list}$)}{}
        $d\coloneqq |x_1-x_2|\mycmt{d\text{ is the minimum distance}}$\\
        $f_1\coloneqq 1,f_2\coloneqq 2\mycmt{(f_1,f_2)\text{ is the target pair}}$\\
        \ForRange{i}{n}{3}{
            \ForRange{j}{n-1}{1}{
                \If{$d>|a_i-a_j|$}{
                    $d=|a_i-a_j|$\\
                    $f_1=i$\\
                    $f_2=j$
                }
            }
        }
        \Return{$(f_1,f_2)$}
    \end{SpacingAlgorithm}

    \ModProcName{greater than the sum of all
        previous terms}
    \mywrongitem \CapProcTargetName \AlgorithmRef
    \begin{SpacingAlgorithm}{\CapProcTargetName}{\CurTodolistToplevelIndex}
        \myproc{\ProcTargetName($\{x_1, x_2,\ldots , x_n\}: \text{a list}$)}{}
        $sum\coloneqq 0\mycmt{sum\text{ is the sum of all
                    previous terms}}$\\
        $S\coloneqq \varnothing\mycmt{S\text{ is the finding term set}}$\\
        \ForRange{i}{2}{n}{
        $sum=sum+x_{i-1}$\\
        \lIf{$sum<x_i$}{
            $\text{append }x_i\text{ to }S$
        }
        }
        \Return{$S$}
    \end{SpacingAlgorithm}
    \ecc{
        \item the ans takes $a_1>0$ in account.
    }

\end{todolist}

\newpage
\begin{todolist}[resume*,start=\CurTodolistToplevelIndex]
    \mywrongitem test 3
    \mywrongitem test 5
    \mywrongitem test 5
    \mywrongitem test 5
\end{todolist}

\end{document}
