% !TeX root = let_infinite_loop.tex

\documentclass{article}

\begin{document}

% https://tex.stackexchange.com/a/4747/308105
\newcommand\foo[1]{\def\arg{#1}\ifx\arg\empty T\else F\fi}
\tableofcontents
\newpage
I originally write this doc to show how to use \verb|\RenewCommandCopy{\Suffix}{_test_}| 
and \verb|\let\Suffix=_test_|

% "TeX will try to expand \def (does nothing), then expand whatever comes next ie the \arg"
% so write the wrong F because \arg isn't run correctly.
\section{Wrong toc \foo{}} % first run writes garbage to the aux file, second crashes
\section{Right toc \protect\foo{}} % this is fine
It is related with \verb|\protect|.\\
\foo{}

% https://tex.stackexchange.com/q/96620/308105
% \escapechar=`^
% \section{\string\let may cause the infinite loop}
\section{\texttt{$\backslash$let} with \texttt{$\backslash$newcommand} as the target
may cause the infinite loop}
% https://tex.stackexchange.com/a/670548/308105
\newcommand\x[1][hmm]{#1!}
\x[A]
% \let would loop 
% and based on this https://tex.stackexchange.com/a/686221/308105
% the kernel may influence the behavior, so just use the kernel offered method better.
% \let\originalx\x
% \show\originalx

% difference from DeclareCommandCopy based on error msg https://tex.stackexchange.com/a/643233/308105
% \DeclareCommandCopy\originalx\x
\NewCommandCopy\originalx\x

\renewcommand\x[1][oops]{\originalx[#1] ??}
% based on the "NewCommandCopy", it will also cause infinite loop,
% so better not use too high level nested command definition.
% \RenewCommandCopy\originalx\x
% \renewcommand\x[1][oops]{\originalx[#1] ??}
% \show\x

\x[B]

\subsection{other examples about \texttt{$\backslash$let} 
and \texttt{$\backslash$protect} making the loop}
% https://tex.stackexchange.com/a/47372/308105
% \let\oribfseries\bfseries
% robust will keep expanding until legal https://texfaq.org/FAQ-protect
% \DeclareRobustCommand\bfseries{\oribfseries\itshape}
% \bfseries sd

\DeclareCommandCopy{\oribfseries}{\bfseries}
\DeclareRobustCommand\bfseries{\oribfseries\itshape}
\bfseries sd

% TODO why this protect no use to make the above loop
\def\test{test}
\let\oribfseries\protect\test
\DeclareRobustCommand\test{\protect\oribfseries}
\test sd

\let\oldemph\emph
\renewcommand{\emph}[1]{\textbf{\oldemph{#1}}}
\emph{23}

% https://texfaq.org/FAQ-activechars
\section{\_ redefinition}
% https://tex.stackexchange.com/a/275975/308105
% \begingroup
\catcode`\_=\active
% \def_{\_}
\def\_{\textunderscore}
% Not use "\NewCommandCopy" with string which will cause the weird behavior, even making the "\catcode`_=8" fail
% \NewCommandCopy{\Suffix}{_3_3}
% \endgroup
\catcode`_=8
% \let can't be used with multiple characters, better with one single cmd.
% \let\test={\textunderscore3\textunderscore\space supplementary}
% \test
\_3\\
% to make the above \_ definition work
\catcode`\_=\active
_3_\\
% \catcode`_=8
\newcommand{\Suffix}{_3}
\NewCommandCopy{\SubSuffix}{\Suffix}
\Suffix\\
\SubSuffix\\

\end{document}